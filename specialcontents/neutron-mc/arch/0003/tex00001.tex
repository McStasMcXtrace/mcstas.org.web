\documentclass[12pt]{report}

\begin{document}
% sections to add in  "manual.tex";
% \section{Detectors and monitors}


\subsection{Monitor\_nD: A general Monitor for 0D/1D/2D records}
\label{s:monitornd}

The component {\bf Monitor\_nD} is a general Monitor that can output any set of 
physical parameters concerning the passing neutrons. The gerenated files are eit
her a set of 1D signals ([Intensity] {\it vs.} [Variable]), or a single 2D signa
l ([Intensity] {\it vs.} [Variable 1] {\it vs.} [Variable 1]), and possibly a si
mple long list of the selected physical parameters for each neutron.

The input parameters for {\bf Monitor\_nD} are its dimensions $x_{\rm min}, x_{\
rm max}, y_{\rm min}$, $y_{\rm max}$ (in meters) and an {\it options} string des
cribing what to detect, and what to do with the signals, in clear language. The 
formatting of the {\it options} parameter is free, as long as it contains some s
pecific keywords, that can be sometimes followed by values. The {\it no} or {\it
 not} option modifier will revert next option. The {\it all} option can also aff
ect a set of monitor configuration parameters (see below). 

\subsubsection{The Monitor\_nD geometry}

The monitor shape can be selected among four geometries:
\begin{enumerate}
\item{({\it square}) The default geometry is flat rectangular in ($xy$) plane wi
th dimensions $x_{\rm min}, x_{\rm max}, y_{\rm min}$, $y_{\rm max}$.}
\item{({\it disk}) When choosing this geometry, the detector is a flat disk in (
$xy$) plane. The radius is then
\begin{equation}
\mbox{radius} =  \max ( \mbox{abs } [ x_{\rm min}, x_{\rm max}, y_{\rm min}, y_{
\rm max} ] ).
\end{equation}
}
\item{({\it sphere}) The detector is a sphere with the same radius as for the {\
it disk} geometry.}
\item{({\it cylinder}) The detector is a cylinder with revolution axis along $y$
 (vertical). The radius in ($xz$) plane is
\begin{equation}
\mbox{radius} =  \max ( \mbox{abs } [ x_{\rm min}, x_{\rm max} ] ),
\end{equation}
and the height along $y$ is 
\begin{equation}
\mbox{height} =  | y_{\rm max} - y_{\rm max} |.
\end{equation}
}
\end{enumerate}

By default, the monitor is flat, rectangular. Of course, you can choose the orie
ntation of the {\bf Monitor\_nD} in the instrument description file with the usu
al \texttt{ROTATED} modifier.

For the {\it sphere} and {\it cylinder}, the incoming neutrons are monitored by 
default, but you can choose to monitor outgoing neutron with the {\it outgoing} 
option.

At last, the {\it slit} or {\it absorb} option will ask the component to absorb 
the neutrons that do not intersect the monitor.

\subsubsection{The neutron parameters that can be monitored}

There are 25 different variables that can be monitored at the same time and posi
tion. Some can have more than one name (e.g. \texttt{energy} or \texttt{omega}).


\begin{verbatim}
 kx ky kz k wavevector (Angs-1) Wavevector norm or coordinates
 vx vy vz v            (m/s)    Velocity norm or coordinates
 x y z radius          (m)      Position and radius in (xy) plane
 t time                (s)      Time of Flight
 energy omega          (meV)    Neutron energy
 lambda wavelength     (Angs)   Neutron wavelength
 p intensity flux      (n/s) or (n/cm^2/s) The neutron weight
 ncounts               (1)      The number of events detected
 sx sy sz              (1)      Spin of the neutron
 vdiv                  (deg)    vertical divergence
 hdiv divergence       (deg)    horizontal divergence
 angle                 (deg)    divergence from <z> direction
 theta longitude       (deg)    longitude (x/z)
 phi   lattitude       (deg)    lattitude (y/z)
\end{verbatim}

To tell the component what you want to monitor, just add the variable names in t
he {\it options} parameter. The data will be sorted into {\it bins} cells (defau
lt is 20), between some default {\it limits}, that can also be set by user. The 
{\it auto} option will automatically determine what limits should be used to hav
e a good sampling of signals.

The {\it with borders} option will monitor variables that are outside the limits
. These values are then accumulated on the 'borders' of the signal.

Each monitoring will record the flux (sum of weights $p$) versus the given varia
bles. The {\it per cm2} option will ask to normalize the flux to the monitor sec
tion surface.

Some examples ?
\begin{enumerate}
\item{\texttt{options="x bins=30 limits=[-0.05 0.05] ; y"} \\
will set the monitor to look at $x$ and $y$. For $y$, default bins and limits va
lues (monitor dimensions) are used.}
\item{\texttt{options="x y, all bins=30, all limits=[-0.05 0.05]"} \\
will do the same, but set limits and bins for $x$ and $y$.}
\item{\texttt{options="x y, auto limits"} \\
will determine itself the required limits for $x$ and $y$ to monitor passing neu
trons with default {\it bins}=20.}
\end{enumerate}

\subsubsection{The output files}

By default, the file names will be the component name, followed by automatic ext
ensions showing what was monitored (such as \texttt{MyMonitor.x}). You can also 
set the filename in {\it options} with the {\it file} keyword followed by the fi
le name that you want. The extension will then be added if the name does not con
tain a dot (.).

The output files format are standard 1D or 2D McStas detector files.
The {\it no file} option will {\it unactivate} monitor, and make it a single 0D 
monitor detecting integrated flux and counts.
The {\it verbose} option will display the nature of the monitor, and the names o
f the generated files.

\subsubsection{The 2D output}

When you ask the {\bf Monitor\_nD} to monitor only two variables (e.g. {\it opti
ons} = "x y"), a single 2D file of intensity versus these two correlated variabl
es will be created.

\subsubsection{The 1D output}

The {\bf Monitor\_nD} can produce a set of 1D files, one for each monitored vari
able, when using 1 or more than 2 variables, or when specifying the {\it multipl
e} keyword option.

\subsubsection{The List output}

The {\bf Monitor\_nD} can additionally produce a {\it list} of variable values f
or neutrons that pass into the monitor. This feature is additive to the 1D or 2D
 output. By default only 1000 events will be recorded in the file, but you can s
pecify for instance "{\it list} 3000 neutrons" or "{\it list all} neutrons". Thi
s last option might require a lot of memory and generate huge files.

\subsubsection{Usage examples}

\begin{itemize}
\item{
\begin{verbatim}
COMPONENT MyMonitor = Monitor_nD( 
    xmin = -0.1, xmax = 0.1, 
    ymin = -0.1, ymax = 0.1, 
    options = "energy auto limits")
\end{verbatim}
will monitor the neutron energy in a single 1D file (a kind of E\_monitor)}
\item{\texttt{{\it options}="x y, all bins=50"} \\
will monitor the neutron $x$ and $y$ in a single 2D file (same as PSD\_monitor)}

\item{\texttt{{\it options}="multiple x bins=30, y limits=[-0.05 0.05]"} \\
will monitor the neutron $x$ and $y$ in two 1D files}
\item{\texttt{{\it options}="x y z kx ky kz,  auto limits"} \\
will monitor theses variables in six 1D files}
\item{\texttt{{\it options}="x y z kx ky kz, list all, auto limits"} \\
will monitor all theses neutron variables in one long list}
\item{\texttt{{\it options}="multiple x y z kx ky kz, and list 2000,  auto limit
s"} \\
will monitor all theses neutron variables in one list of 2000 events and in six 
1D files}
\end{itemize}


% \section{Sources}

\subsection{Source\_Optimizer: A general Optimizer for McStas}
\label{s:sourceoptimizer}

The component {\bf Source\_Optimizer} optimizes the whole neutron flux in order 
to achieve better statistics at each {\bf Monitor\_Optimizer} location(s) (see s
ection (\ref{s:monitoroptimizer}) for this latter component. It can act on any i
ncoming neutron beam (from any source type), and more than one optimization crit
eria location can be placed along the instrument. 

The usage of the optimizer is very simple, and usually does not require any conf
iguration parameter. Anyway the user can still customize the optimization {\it v
ia} various {\it options}.

The optimizer efficiency makes it easy to increase the number of events at optim
ization criteria locations by a factor of 20, and thus decreases the signal erro
r bars by a factor 4.5. Higher factors can often be achieved in practise. Of cou
rse, the overall flux remains the same as without optimizer.

\subsubsection{The optimization algorithm}

When a neutron reaches the {\bf Monitor\_Optimizer} location(s), the component r
ecords what were its position ($x$, $y$) and speed ($v_x, v_y, v_z$) when it pas
sed in the {\bf Source\_Optimizer}. Some distribution tables of {\it good} neutr
ons characteristics are then built. 

When a {\it bad} neutron comes to the {\bf Source\_Optimizer} (it would then hav
e few chances to reach {\bf Monitor\_Optimizer}), it is changed into a better on
e. That means that its position and velocity coordinates are translated to bette
r values according to the {\it good} neutrons distribution tables. Anyway, the n
eutron energy ($\surd v_x^2 + v_y^2 + v_z^2$) is kept as far as possible. 

The {\bf Source\_Optimizer} works as follow:
\begin{enumerate}
\item{First of all, the {\bf Source\_Optimizer} determines some limits ({\it min
} and {\it max}) for variables $x, y, v_x, v_y, v_z$.}
\item{Then the component records the non-optimized flux distributions in arrays 
with {\it bins} cells (default is 10 cells). This constitutes the {\it Reference
} source.}
\item{\label {SourceOptimizer:step3}The {\bf Monitor\_Optimizer} records the {\i
t good} neutrons (that reach it) and communicate an {\it Optimized} source to th
e {\bf Source\_Optimizer}. However, '{\it keep}' percent of the original {\it Re
ference} source is sent unmodified (default is 10 \%). The {\it Optimized} sourc
e is thus: 

\begin{center}
\begin{tabular}{rcl}
{\it Optimized} & = & {\it keep} * {\it Reference} \\
& + & (1 - {\it keep}) [Neutrons that will reach monitor].
\end{tabular}
\end{center}
}
\item{The {\bf Source\_Optimizer} transforms the {\it bad} neutrons into {\it go
od} ones from the {\it Optimized} source. 
The resulting optimised flux is normalised to the non-optimized one:
\begin{equation}
p_{optimized} = p_{initial} \frac{\mbox{Reference}}{\mbox{Optimized}},
\end{equation}
and thus the overall flux at {\bf Monitor\_Optimizer} location is the same as wi
thout the optimizer. Usually, the process sends more {\it good} neutrons from th
e {\it Optimized} source than than in the {\it Reference} one.
The energy (and velocity) spectra of neutron beam is also kept, as far as possib
le. For instance, an optimization of $v_z$ will induce a modification of $v_x$ o
r $v_y$ to try to keep $|\vec{v}|$ constant.
}
\item{When the {\it continuous} optimization option is activated (by default), t
he process loops to Step (\ref{SourceOptimizer:step3}) every '{\it step}' percen
t of the simulation. This parameter is computed automatically (usually around 10
 \%) in {\it auto} mode, but can also be set by user.}
\end{enumerate}

During steps (1) and (2), some non-optimized neutrons with original weight $p_{i
nitial}$ may  lead to spikes on detector signals. This is greatly improved by lo
wering the weight $p$ during these steps, with the {\it smooth} option.
The component optimizes the neutron parameters on the basis of independant varia
bles. Howver, it usually does work fine when these variables are correlated (whi
ch is often the case in the course of the instrument simulation).
The memory requirements of the component are very low, as no big $n$-dimensional
 array is needed.

\subsubsection{Using the Source\_Optimizer}

To use this component, just install the {\bf Source\_Optimizer} after a source (
but any location is possible in principle), and use the {\bf Monitor\_Optimizer}
 at a location where you want to have better statistics.

\begin{verbatim}
    /* where to act on neutrons */
    COMPONENT optim_s = Source_Optimizer(options="") 
    ...
    /* where to have better statistics */
    COMPONENT optim_m = Monitor_Optimizer( 
    xmin = -0.05, xmax = 0.05, 
    ymin = -0.05, ymax = 0.05,
    optim_comp = optim_s) 
    ...
    /* using more than one Monitor_Optimizer is possible */
\end{verbatim}

The input parameter for {\bf Source\_Optimizer} is a single {\it options} string
 that can contain some specific optimizer configuration settings in clear langua
ge. The formatting of the {\it options} parameter is free, as long as it contain
s some specific keywords, that can be sometimes followed by values.

The default configuration (equivalent to {\it options} = "") is
\begin{center}
\begin{tabular}{rcl}
{\it options} & = & "{\it continuous} optimization, {\it auto} setting, {\it kee
p} = 10, {\it bins} = 10, \\
& & {\it smooth} spikes, and do {\it not free} energy during optimization".
\end{tabular}
\end{center}
The keyword modifiers {\it no} or {\it not} revert the next option. Other option
s not shown here are:
\begin{verbatim}
verbose         displays optimization process (debug purpose).
unactivate      to unactivate the Optimizer.
file=[name]     Filename where to save optimized source distributions
\end{verbatim}
The {\it file} option will save the source distributions at the end of the optim
ization. If no name is given the component name will be used, and a '.src' exten
sion will be added. By default, no file is generated. The file format is in a Mc
Stas 2D record style.

\subsection{Monitor\_Optimizer: Optimization locations for the Source\_Optimizer
 component}
\label{s:monitoroptimizer}

The {\bf Monitor\_Optimizer} component works with the {\bf Source\_Optimizer} co
mponent. See section (\ref{s:sourceoptimizer}) for usage.

The input parameters for {\bf Monitor\_Optimizer} are the rectangular shaped ope
ning coordinates $x_{\rm min}, x_{\rm max}, y_{\rm min}$, $y_{\rm max}$ (in mete
rs), and the name of the associated {\bf Source\_Optimizer} used in the instrume
nt description file (one word, without quotes).

\end{document}
